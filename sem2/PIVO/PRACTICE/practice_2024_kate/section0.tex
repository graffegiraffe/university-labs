%Пример

\begin{SCn}
\begin{small}

\scnheader{Часть 3 Учебной дисциплины "Представление и обработка информации в интеллектуальных системах"{}}
\begin{scnrelfromlist}{библиографическая ссылка}
    \scnitem{Технология комплексной поддержки жизненного цикла семантически совместимых интеллектуальных компьютерных систем нового поколения}
        \begin{scnindent}
        \scntext{URL}{https://libeldoc.bsuir.by/handle/123456789/51151}
    \end{scnindent} 
    \scnitem{Приобретение знаний интеллектуальными системами}
      \begin{scnindent}
        \scntext{URL}{https://elibrary.ru/item.asp?id=19690065}
    \end{scnindent}
    \scnitem{Гибридные интеллектуальные системы. Теория и технология разработки}
    \begin{scnindent}
        \scntext{URL}{https://www.dissercat.com/content/tekhnologiya-razrabotki-gibridnykh-intellektualnykh-sistem}
    \end{scnindent}
    \scnitem{Классификация}
    \begin{scnindent}
        \scntext{URL}{https://science-education.ru/ru/article/view?id=16963}
    \end{scnindent}
        \scnitem{Volume I Basic Programming Guide}
    \begin{scnindent}
        \scntext{URL}{http://crowley-coutaz.fr/jlc/Courses/2018/MoSIG.SIRR/bpg63.pdf}
    \end{scnindent}
            \scnitem{Metasystem of the OSTIS Technology and the Standard of the OSTIS Technology}
    \begin{scnindent}
        \scntext{URL}{https://libeldoc.bsuir.by/bitstream/123456789/49330/Metasystem.pdf}
    \end{scnindent}


\end{scnrelfromlist}

\begin{scnrelfromvector}{аттестационные вопросы}
    \scnitem{Вопрос 1 по Части 3 Учебной дисциплины "Представление и обработка информации в интеллектуальных системах"{}}
    \scnitem{Вопрос 2 по Части 3 Учебной дисциплины "Представление и обработка информации в интеллектуальных системах"{}}
    \scnitem{Вопрос 3 по Части 3 Учебной дисциплины "Представление и обработка информации в интеллектуальных системах"{}}
    \scnitem{Вопрос 4 по Части 3 Учебной дисциплины "Представление и обработка информации в интеллектуальных системах"{}}

\end{scnrelfromvector}

\scnheader{Вопрос 1 по Части 3 Учебной дисциплины "Представление и обработка информации в интеллектуальных системах"{}}
\scnidtf{Операционная семантика логических языков. Предметная область логических моделий решения задач. Абстракный sc-агент}
\begin{scnrelfromlist}{библиографическая ссылка}
    \scnitem{Голенков В.В..ТехКомпПодЖЦССИКСНП-2023art}

    \begin{scnindent}
         \scnidtf{Технология комплексной поддержки жизненного цикла семантически совместимых интеллектуальных компьютерных систем нового поколения}
    \end{scnindent}
\end{scnrelfromlist}

\scnheader{Вопрос 2 по Части 3 Учебной дисциплины "Представление и обработка информации в интеллектуальных системах"{}}
\scnidtf{Решение вопроса совместимости искусственных нейронных сетей}
\begin{scnrelfromlist}{библиографическая ссылка}
    \scnitem{Голенков В.В..ТехКомпПодЖЦССИКСНП-2023art}

    \begin{scnindent}
         \scnidtf{Технология комплексной поддержки жизненного цикла семантически совместимых интеллектуальных компьютерных систем нового поколения}
    \end{scnindent}
\end{scnrelfromlist}

\scnheader{Вопрос 3 по Части 3 Учебной дисциплины "Представление и обработка информации в интеллектуальных системах"{}}
\scnidtf{Языки продукционного программирования}
\begin{scnrelfromlist}{библиографическая ссылка}
    \scnitem{Голенков В.В..ТехКомпПодЖЦССИКСНП-2023art}

    \begin{scnindent}
         \scnidtf{Технология комплексной поддержки жизненного цикла семантически совместимых интеллектуальных компьютерных систем нового поколения}
    \end{scnindent}

\scnitem{Beta, Quicksilver.. Vol I BPG-2008art}
    \begin{scnindent}
         \scnidtf{Volume I Basic Programming Guide}
    \end{scnindent}
   
\scnitem{Гаврилов А.В.. ГибридИС-2009art}
    \begin{scnindent}
         \scnidtf{Гибридные интеллектуальные системы}
    \end{scnindent}

\scnitem{Колесников А.В... ГибридИС.ТиТР-2008art}
    \begin{scnindent}
         \scnidtf{Гибридные интеллектуальные системы. Теория и технология разработки}
    \end{scnindent}

 \scnitem{Субботин А.Л... Классиф-2001art}
    \begin{scnindent}
         \scnidtf{Классификация}
    \end{scnindent}
    
\end{scnrelfromlist}

\scnheader{Вопрос 4 по Части 3 Учебной дисциплины "Представление и обработка информации в интеллектуальных системах"{}}
\scnidtf{Операционная семантика моделей искуственных нейронных сетей}
\begin{scnrelfromlist}{библиографическая ссылка}
    \scnitem{Голенков В.В..ТехКомпПодЖЦССИКСНП-2023art}

    \begin{scnindent}
         \scnidtf{Технология комплексной поддержки жизненного цикла семантически совместимых интеллектуальных компьютерных систем нового поколения}
    \end{scnindent}

      \scnitem{Осипов Г.С.. ПриобЗИС-2012art}
    \begin{scnindent}
         \scnidtf{Приобретение знаний интеллектуальными системами}
    \end{scnindent}
        
  \scnitem{Bantsevich K.A.Metas otOSTIS-2022art}
    \begin{scnindent}
         \scnidtf{Metasystem of the OSTIS Technology and the Standard
of the OSTIS Technology}
    \end{scnindent}
\end{scnrelfromlist}


\end{small}
\end{SCn}
