\begin{SCn}
\begin{small}

\scnheader{OPS5}
\scnidtf{Полноценный язык программирования для продукционного программирования}
\begin{scnrelfromset}{декомпозиция}
            \scnitem{MAKE}
            \scntext{пояснение}{Создает новый элемент рабочей памяти.}
            \scnitem{MODIFY}
            \scntext{пояснение}{Изменяет один или несколько значений атрибутов у существующего элемента рабочей памяти.}
            \scnitem{REMOVE}
             \scntext{пояснение}{Удаляет элемент рабочей памяти.}
             
           \scnrelfrom{изображение}{\scnfileimage[26em]{images/123.png}}
          \scntext{пояснение}{Пример правила продукции на OPS5.}


\end{scnrelfromset}

\scnheader{CLIPS}
\scnidtf{Программная среда для разработки экспертных систем}
\scnidtf{Средство разработки экспертных систем, база знаний которых представляет совокупность правил продукции}
\scntext{пояснение}{CLIPS использует продукционную модель представления знаний и поэтому содержит три основных элемента.}
\begin{scnrelfromset}{разбиение}
    \scnitem{базу фактов (fact base)}
    \scnitem{базу правил (rule base)}
    \scnitem{механизм логического вывода}
\end{scnrelfromset}

\scnheader{Факты}
\scnidtf{Одна из основных форм представления информации в системе CLIPS}
\scntext{пояснение}{Каждый факт представляет фрагмент информации, который был помещен в текущий список фактов, называемый fact-list. Факт представляет собой основную единицу данных, используемую правилами.}

\scnheader{Идентификатор факта }
\scnidtf{Это короткая запись для отображения факта на экране}

\scnheader{Алгоритм Rete}
\scnidtf{Содержит обобщение логики функционала, ответственного за связь данных (фактов) и алгоритма (продукций) в системах сопоставления с образцом (вид систем: системы основанные на правилах)}
\begin{scnrelfromset}{обобщенная декомпозиция}
    \scnitem{Алгоритм Rete имеет следующие характеристики:}
        \scntext{пояснение}{CLIPS использует продукционную модель представления знаний и поэтому содержит три основных элемента.}
        \scntext{пояснение}{Сохраняет частичные соответствия между фактами при слиянии разных типов фактов. Это позволяет избежать полного вычисления всех фактов при любом изменении в рабочей памяти продукционной системы. Система работает только с самими изменениям.}
        \scntext{пояснение}{Сохраняет частичные соответствия между фактами при слиянии разных типов фактов. Это позволяет избежать полного вычисления всех фактов при любом изменении в рабочей памяти продукционной системы. Система работает только с самими изменениям.}
        \scntext{пояснение}{Позволяет эффективно высвобождать память при удалении фактов.}
\end{scnrelfromset}

\scnheader{Миварный подход}
\scnidtf{Математический аппарат для разработки систем искусственного интеллекта, созданный путем комплексирования продукционного подхода}
\begin{scnrelfromset}{обобщенная декомпозиция}
    \scnitem{Миварный подход объединяет две основные технологии накопления данных и обработки информации:}
        \scntext{примечание}{Миварное информационное пространство: накопление данных на основе эволюционной самоорганизующейся миварной модели данных с изменяющейся структурой в теории баз данных.}
        \scntext{примечание}{Миварные сети: обработка информации на основе развития продукционного подхода к логическому выводу с учетом включения возможности автоматического конструирования алгоритмов для "решателей задач" и традиционной вычислительной обработки, а также с использованием идей отношений, правил и процедур, которые теперь принято относить к сервисно-ориентированным архитектурам и многоагентным системам.}
    
\end{scnrelfromset}






\end{small}
\end{SCn}