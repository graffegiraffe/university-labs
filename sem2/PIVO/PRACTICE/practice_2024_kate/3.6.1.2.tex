\begin{SCn}
\begin{small}

\scnheader{Операционная семантика языка представления нейросетевого метода в базах знаний }
\scnidtf{Задается многоагентный подход к интерпретации искусственных нейронных сетей и спецификацией соответствующих действий}

\scnheader{Действие интерпретации слоя и.н.с.}

\begin{scnrelfromset}{декомпозиция}
    \scnitem{действие вычисления взвешенной суммы всех нейронов слоя}
    \scnitem{действие вычисления функции активации всех нейронов слоя}
    \scnitem{действие интерпретации сверточного слоя}
    \scnitem{действие интерпретации пулинг слоя}
          \scnrelfrom{изображение}{\scnfileimage[25em]{images/124.png}}
          \scntext{пояснение}{SCg-текст. Пример формализации архитектуры искусственной нейронной сети в базе знаний.}

\end{scnrelfromset}

\scnheader{Действие вычисления взвешенной суммы всех нейронов слоя}
\begin{scnrelfromlist}{Аргументы(объекты’) этого действия задаются следующими отношениями:}
    \scnitem{\textbf{входной вектор’}}
         \scnrelfrom{первый домен}{действие интерпретации и.н.с.}
         \scnrelfrom{второй домен}{ориентированное множество чисел}
    \scnitem{\textbf{матрица весовых коэффициентов нейронов слоя’}}
        \scnrelfrom{первый домен}{ действие по обработке и.н.с.}
         \scnrelfrom{второй домен}{матрица}
         \scntext{пояснение}{Результатом является ориентированное множество чисел, являющихся взвешенной суммой  нейронов соответствующего слоя.}
\end{scnrelfromlist}

\scnheader{Действие вычисления функции активации всех нейронов слоя}
\begin{scnrelfromlist}{Аргументы этого действия задаются следующими отношениями:}
    \scnitem{\textbf{вектор взвешенных сумм нейронов слоя’}}
         \scnrelfrom{первый домен}{действие по обработке и.н.с.}
         \scnrelfrom{второй домен}{ ориентированное множество чисел}
          \scnrelfrom{изображение}{\scnfileimage[26em]{images/125.png}}
          \scntext{пояснение}{SCg-текст. Пример действия вычисления взвешенной суммы всех нейронов слоя.}
 
    \scnitem{\textbf{вектор порогов нейронов слоя’}}
        \scnrelfrom{первый домен}{ действие по обработке и.н.с.}
         \scnrelfrom{второй домен}{ориентированное множество чисел}
    \scnitem{\textbf{функция активации’}}
        \scnrelfrom{первый домен}{ действие по обработке и.н.с.}
         \scnrelfrom{второй домен}{ функция}
         \scntext{пояснение}{Результатом действия является ориентированное множество чисел, являющихся выходными значениями нейронов слоя.}
\end{scnrelfromlist}

\scnheader{Действие интерпретации сверточного слоя}
\begin{scnrelfromlist}{Аргументы этого действия задаются следующими отношениями:}
    \scnitem{\textbf{входная матрица}’}
         \scnrelfrom{первый домен}{действие интерпретации и.н.с.}
         \scnrelfrom{второй домен}{матрица}
    \scnitem{\textbf{ядро свертки’}}
        \scnrelfrom{первый домен}{ действие интерпретации сверточного слоя}
         \scnrelfrom{второй домен}{ число}
    \scnitem{\textbf{шаг свертки’}}
        \scnrelfrom{первый домен}{ действие интерпретации сверточного слоя}
         \scnrelfrom{второй домен}{число}
         \scntext{пояснение}{Результатом действия является матрица, полученная в результате свертки входной матрицы с ядром свертки.}
\end{scnrelfromlist}

\scnheader{Действие интерпретации пулинг слоя}
\begin{scnrelfromlist}{Аргументы этого действия задаются следующими отношениями:}
    \scnitem{\textbf{входная матрица’}}
         \scnrelfrom{первый домен}{действие интерпретации и.н.с.}
         \scnrelfrom{второй домен}{матрица}
    \scnitem{\textbf{размер окна пулинга’}}
        \scnrelfrom{первый домен}{ действие интерпретации пулинг слоя}
         \scnrelfrom{второй домен}{ матрица}
    \scnitem{\textbf{шаг окна пулинга’}}
        \scnrelfrom{первый домен}{ действие интерпретации пулинг слоя}
         \scnrelfrom{второй домен}{число}
         \scntext{пояснение}{Результатом действия является матрица, полученная в результате пулинга входной матрицы.}
\end{scnrelfromlist}

\scnheader{Предметная область нейросетевых методов}
\scnidtf{Предметная область искусственных нейронных сетей}
\begin{scnrelfromset}{дочерняя предметная область}
    \scnitem{Предметная область нейросетевых методов SCP}
    \scnitem{Предметная область нейросетевых методов Python}
    \scnitem{Предметная область нейросетевых методов C++}
\end{scnrelfromset}

\scnheader{Ориентированное множество чисел}
\scnidtf{Ормножество чисел}
\scnrelto {включение}{число} 
\scnrelto {включение}{ориентированное множество}
\scnrelto {первый домен}{строковое представление ормножества чисел*}
\scntext{пояснение}{Матрица является ориентированным множеством ориентированных множеств чисел равной мощности.}

\scnheader{Нейросетевой метод}
\scnidtf{Искусственная нейронная сеть}
\scntext{пояснение}{Предлагается рассматривать и.н.с. как класс методов решения задач со своим языком представления. В соответствии с Технологией OSTIS, спецификация класса методов решения задач сводится к спецификации соответствующего языка представления методов, то есть к описанию его синтаксической, денотационной и операционной семантики.}
    \scnrelfrom{изображение}{\scnfileimage[20em]{images/126.png}}
          \scntext{пояснение}{Решение задачи ИСКЛЮЧАЮЩЕЕ ИЛИ.}

              \scnrelfrom{изображение}{\scnfileimage[20em]{images/127.png}}
          \scntext{пояснение}{Схема однослойного персептрона, решающего задачу ИСКЛЮЧАЮЩЕЕ ИЛИ.}
     \scnrelfrom{изображение}{\scnfileimage[30em]{images/128.png}}
          \scntext{пояснение}{Метод, решающий задачу ИСКЛЮЧАЮЩЕЕ ИЛИ, представленный с помощью языка представления нейросетевых методов SCP.}


\end{small}
\end{SCn}